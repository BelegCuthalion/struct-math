\section{Introduction}

\subsection*{}
\emph{``If a straight line meets another one, then we can not find another straight line which does not meet both of them.''}
This proposition is a simple consequence of what we know as Euclidean geometry, where we talk about ``straight line'' and ``meets'' as undefined, primitive notions.
We can show that it is \emph{true}, using Euclid's axioms and rules of logic. We can also conclude and state more facts about these geometry, using axioms, and construct geometry in a synthetic way.
But what is this proposition talking about? How are we interpreting abstract notions such as a ``straight line'' or ``meets'' that makes the proposition true?

The axioms, and all propositions derived this way, are too abstract to tell us anything, but there are also \emph{realizations} of Euclidean geometry where these questions have answers.
We all know one such realization as analytic geometry, where a straight line is \emph{represented} by a first-degree polynomial in $\mathbb{R}^2$, and we say two such lines meet when the difference of their corresponding polynomials is $0$.
In this sense, analytic geometry tells us the same facts as the Euclid's axioms, but they are reduced to more tangible, familiar notions, and stated in a richer language. So for example we can also use our knowledge about the real numbers and their arithmetic while in fact talking about points and lines.

This is what one might call an \emph{analytic} approach to something, (geometry here) versus the former \emph{synthetic} approach.
In a synthetic approach to a problem we would build up the solution from a basis, like a set of axioms.
In contrast, an analytic approach to a problem looks for a solution in what we already know.
And going from the synthetic approach to the analytic one is simple as finding the proper \emph{representation} in a richer universe.
As for the example here, in analytic geometry, abstract notion of a point on a plane is represented by a pair of real numbers, we can extend all other notions from here.
We can find other examples of this analytic-synthetic duality in mathematics, anywhere that some abstract notion is reduced to a another, more conrete one.
For example when the theory of natural numbers, PA, is realized in set theory, and the natural numbers are represented by Von-Neumann's ordinals.

\subsection*{A representation problem}
As much as our language got richer, more, sometimes irrelevant notions show up.
Now another questions may arise: Isn't this new universe a little too much rich for Euclidean geometry?
For example, it does not tell anything about coordinations.
In proving something like the mentioned proposition, coordinations on the plane are irrelevant to the proof, since there is no such notion in the Euclidean geometry at all.
Nevertheless, coordination is the primitive notion of our analytic universe and we must carry it all the way in our proof, introducing unnecessary complexity to our work.
So does it make sense to speak about coordinations of a geometrical object, while we can move the origin on the plane and still prove the same geometrical facts?

It seems we have a \emph{representation problem}; we have different objects representing the same thing. This problem is not ad hoc to this example of Euclidean geometry here, since it is an inevitable by-product of our analytic approach.
We are moving to a richer universe after all, and facing issues native to that univers, but irrelevant to our problem, would not be unexpected.

To handle this problem, we must define some ``equivalence'' between these different representations.
In our example, this equivalence is Euclidean trasnformation, i.e. we will equalize any representations if they can be transformed to each other.
So the representations of geometric objects are not the subsets of $\mathbb{R}^2$, but equivalence classes of them, up to Euclidean trasnformations.

Now we have more clear idea about what a geometrical object or a geometrical property is. We can think of a geometrical object as these equivalence classes and call any property that is held by trasnformations as ``geometrical'', because such properties are thats of the geometrical objects, and not their representations.
So it is clear that, for example, we will not consider coordinations ``geometrical''.

But while it seems we are studying the correct objects now, one issue is still in place, if not worsen: We are imposing even more complexity here, because now instead of coordinations we must carry around the whole machinary of equivalence classes.
This might seem unproblematic in simple cases; we can always ``forget'' this machinary and act if the equivalence classes are just abstract ``geometrical objects''.
Even if it would be this easy, for in some cases it is not, it sounds like returning back to where we just started, i.e. abstract notions.
Moreover, in many examples of analytic approach this complexity overload will haunt you as long as you are developing the theory and will prevent you from thinking clearly about the correct object of study.


\subsection*{A synthetic approach, from without}

